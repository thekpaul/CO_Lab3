\documentclass[10pt]{article}
\usepackage[margin=.75in, bottom=1in]{geometry}
\usepackage[english]{babel}
% \usepackage[newfloat]{minted} -> Unused for subjects without code
\usepackage[doublespacing]{setspace}
\usepackage[utf8]{inputenc}
\usepackage[dvipsnames]{xcolor}
\usepackage[tbtags]{amsmath}
\usepackage{kotex, amssymb, amsfonts, mathtools, pgf, enumitem, xfrac, siunitx,
  etoolbox, array, pgfplots, caption, esint, float, tikz, standalone, graphicx,
  ifthen, fancyhdr, titletoc, titlesec, multirow, multicol, tgpagella, bm}
\DeclareUnicodeCharacter{2212}{−}
\usepgfplotslibrary{groupplots,dateplot}
\usetikzlibrary{angles, quotes, patterns,shapes.arrows}
\pgfplotsset{compat=1.18}
\usepackage[hidelinks]{hyperref}
\pagestyle{fancy}
  \fancyhf{}
  \fancyhead[L]{Lab 3 : Branch Predictor Integration}
  \fancyhead[R]{\thepage}
  \fancyfoot[L]{2022년 1학기 컴퓨터조직론}
  \fancyfoot[R]{전기·정보공학부 2018-17515 김형준}
  \renewcommand{\headrulewidth}{0.5pt}
  \renewcommand{\footrulewidth}{0.5pt}
\renewcommand{\bar}[1]{%
  \mkern 1.5mu\overline{%
    \mkern 1.5mu #1 \mkern 1.5mu%
  } \mkern 1.5mu%
} \setlength{\parindent}{0.2in}
\newcolumntype{C}[1]{>{\centering\arraybackslash}p{#1}}
\newcolumntype{R}[1]{>{\raggedleft\arraybackslash}p{#1}}
\newcolumntype{M}[1]{>{\centering\arraybackslash}m{#1}}
% \setlength\tabcolsep{0pt}
\newcommand*\circled[1]{\tikz[baseline=(char.base)]{
  \node[shape=circle,draw,inner sep=0.5pt] (char) {\footnotesize{#1}};}
}
\newcommand{\qtyt}{\qty[mode = text]}
\newcommand{\numt}{\num[mode = text]}
\AtBeginDocument{%
  \setlength\abovedisplayskip{5pt}
  \setlength\belowdisplayskip{5pt}
  \setlength\abovedisplayshortskip{3pt}
  \setlength\belowdisplayshortskip{3pt}
} \sisetup{inter-unit-product = \ensuremath{{\!}\cdot{\!}}, per-mode = symbol}
\setlist[enumerate,1]{label=\arabic*.,leftmargin=0.2in}
\setlist[enumerate,2]{label=\arabic*),leftmargin=0.2in}
\renewcommand{\eqref}[1]{식 \ref{#1}}
\newcommand{\kohm}[1]{\qty{#1}{\kilo\ohm}}
\NewDocumentCommand \widecaps { o o m o } {%
  \IfNoValueTF{#2} {%
    \parbox{\linewidth}{%
      \IfNoValueTF{#1} { \captionof{figure}{#3} } { \captionof{#1}{#3} }%
      \IfValueTF{#4} { \label{#4} } {}%
    }%
  } {%
    \parbox{#1}{%
      \IfNoValueTF{#2} { \captionof{figure}{#3} } { \captionof{#2}{#3} }%
      \IfValueTF{#4} { \label{#4} } {}%
    }%
  }%
}
% \captionsetup[figure]{aboveskip=5pt,belowskip=\baselineskip}
% If-and-only-if Notation
\NewDocumentCommand \iaoi { o } {%
  \IfNoValueTF{#1} { \quad \Leftrightarrow \quad } { #1 \Leftrightarrow #1 }%
}
\makeatletter
\let\mytagform@=\tagform@
\def\tagform@#1{\maketag@@@{\scshape(\ignorespaces#1\unskip\@@italiccorr)}%
    \hspace{3mm}}
\makeatother
\setlength{\columnsep}{0.3in}
\setlength{\columnseprule}{0.2pt}
% MATH
% Better Vector Notation
\makeatletter
\newlength\xvec@height%
\newlength\xvec@depth%
\newlength\xvec@width%
\newcommand{\xvec}[2][]{%
  \ifmmode%
    \settoheight{\xvec@height}{$#2$}%
    \settodepth{\xvec@depth}{$#2$}%
    \settowidth{\xvec@width}{$#2$}%
  \else%
    \settoheight{\xvec@height}{#2}%
    \settodepth{\xvec@depth}{#2}%
    \settowidth{\xvec@width}{#2}%
  \fi%
  \def\xvec@arg{#1}%
  \def\xvec@dd{:}%
  \def\xvec@d{.}%
  \raisebox{.2ex}{\raisebox{\xvec@height}{\rlap{%
    \kern.05em%  (Because left edge of drawing is at .05em)
    \begin{tikzpicture}[scale=1]
    \pgfsetroundcap
    \draw (.05em,0)--(\xvec@width-.05em,0);
    \draw (\xvec@width-.05em,0)--(\xvec@width-.15em, .075em);
    \draw (\xvec@width-.05em,0)--(\xvec@width-.15em,-.075em);
    \ifx\xvec@arg\xvec@d%
      \fill(\xvec@width*.45,.5ex) circle (.5pt);%
    \else\ifx\xvec@arg\xvec@dd%
      \fill(\xvec@width*.30,.5ex) circle (.5pt);%
      \fill(\xvec@width*.65,.5ex) circle (.5pt);%
    \fi\fi%
    \end{tikzpicture}%
  }}}%
  #2%
}
\makeatother

\let\stdvec\vec
% \renewcommand{\vec}[1]{\xvec[]{#1}}

% --- Define \dvec and \ddvec for dotted and double-dotted vectors.
\newcommand{\dvec}[1]{\xvec[.]{#1}}
\newcommand{\ddvec}[1]{\xvec[:]{#1}}

% Personalised macros
\DeclareRobustCommand{\[}{\begin{equation}}
\DeclareRobustCommand{\]}{\end{equation}}
\let\myvec\vec
\renewcommand{\vec}[1]{\mathbf{#1}} % Vector with varying height
\newcommand{\hivec}[2][b]{\xvec[]{\vphantom{#1}#2}} % Vector with consistent height
\newcommand{\abs}[1]{\ensuremath{\lvert #1 \rvert}}
\newcommand{\round}[1]{\ensuremath{\partial #1}}
\newcommand{\D}[1]{\ensuremath{\Delta #1}}
\NewDocumentCommand \len { o m } {
  \IfNoValueTF{#1} { \abs{\vec{#2}} } { \abs{\xvec[]{\vphantom{#1}#2}} }
}
\newcommand{\mat}[1]{\mathbf{#1}} % Matrix Notation
\NewDocumentCommand \pafr { o m o m o } {%
  \IfNoValueTF{#5} {%
    \IfNoValueTF{#1} {%
      \IfNoValueTF{#3} {%
        \ensuremath{\frac{\round{#2}}{\round{#4}}}
      } {%
        \ensuremath{\frac{\round{#2}}{#3\,\round{#4}}}
      }
    } {%
      \IfNoValueTF{#3} {%
        \ensuremath{\frac{#1\,\round{#2}}{\round{#4}}}
      } {%
        \ensuremath{\frac{#1\,\round{#2}}{#3\,\round{#4}}}
      }
    }
  } {
    \IfNoValueTF{#1} {%
      \IfNoValueTF{#3} {%
        \ensuremath{\frac{\partial\sp{#5}#2}{\partial #4\sp{#5}}}
      } {%
        \ensuremath{\frac{\partial\sp{#5}#2}{#3\,\partial #4\sp{#5}}}
      }
    } {%
      \IfNoValueTF{#5} {%
        \ensuremath{\frac{#1\,\partial\sp{#5}#2}{\partial #4\sp{#5}}}
      } {%
        \ensuremath{\frac{#1\,\partial\sp{#5}#2}{#3\,\partial #4\sp{#5}}}
      }
    }
  }
}
\NewDocumentCommand \difr { o m o m o } {%
  \IfNoValueTF{#5} {%
    \IfNoValueTF{#1} {%
      \IfNoValueTF{#3} {%
        \ensuremath{\frac{d #2}{d #4}}
      } {%
        \ensuremath{\frac{d #2}{#3\,d #4}}
      }
    } {%
      \IfNoValueTF{#3} {%
        \ensuremath{\frac{#1\,d #2}{d #4}}
      } {%
        \ensuremath{\frac{#1\,d #2}{#3\,d #4}}
      }
    }
  } {
    \IfNoValueTF{#1} {%
      \IfNoValueTF{#3} {%
        \ensuremath{\frac{d\sp{#5} #2}{d #4\sp{#5}}}
      } {%
        \ensuremath{\frac{d\sp{#5} #2}{#3\,d #4\sp{#5}}}
      }
    } {%
      \IfNoValueTF{#5} {%
        \ensuremath{\frac{#1\,d\sp{#5} #2}{d #4\sp{#5}}}
      } {%
        \ensuremath{\frac{#1\,d\sp{#5} #2}{#3\,d #4\sp{#5}}}
      }
    }
  }
}
\NewDocumentCommand \VTP { o m m m } {%
  \IfNoValueTF{#1} {%
    \ensuremath{\vec{#2}\times\vec{#3}\times\vec{#4}}
  } {
    \ensuremath{\hivec[#1]{#2}\times\hivec[#1]{#3}\times\hivec[#1]{#4}}
  }
}
\NewDocumentCommand \STP { o m m m } {%
  \IfNoValueTF{#1} {%
    \ensuremath{\vec{#2}\wedge\vec{#3}\wedge\vec{#4}}
  } {
    \ensuremath{\hivec[#1]{#2}\wedge\hivec[#1]{#3}\wedge\hivec[#1]{#4}}
  }
}
\newcommand{\vecf}[2]{\ensuremath\vec{#1}\left(#2\right)}
\newcommand{\grad}[1]{\ensuremath\nabla #1}
\newcommand{\dive}[1]{\ensuremath\nabla\DOT{#1}}
\newcommand{\curl}[1]{\ensuremath\nabla\times{#1}}
\newcommand{\lapl}[1]{\ensuremath\nabla\sp{2}#1}
\makeatletter
\newcommand*{\DOT}{}% Check if undefined
\DeclareRobustCommand*{\DOT}{%
  \mathbin{\mathpalette\DOT@{}}%
}
\newcommand*{\DOT@scalefactor}{.5}
\newcommand*{\DOT@widthfactor}{1.15}
\newcommand*{\DOT@}[2]{%
  % #1: math style
  % #2: unused
  \sbox0{$#1\vcenter{}$}% math axis
  \sbox2{$#1\cdot\m@th$}%
  \hbox to \DOT@widthfactor\wd2{%
  \hfil
  \raise\ht0\hbox{%
    \scalebox{\DOT@scalefactor}{%
    \lower\ht0\hbox{$#1\bullet\m@th$}%
    }%
  }%
  \hfil
  }%
}
\makeatother
\NewDocumentCommand \uvec { o } {%
  \IfValueTF{#1}{\ensuremath{\vec{a}\sb{#1}}}%
    {\ensuremath{\vec{\mathrm{a}}}}
}

% % CODE
% \NewDocumentCommand \mi { o } {%
%   \IfValueTF{#1} {%
%     \mintinline[#1]{<++>}%
%   } {%
%     \mintinline{<++>}%
%   }%
% }
%
% \SetupFloatingEnvironment{listing}{name=Code}
% \captionsetup[listing]{aboveskip=0pt,belowskip=0pt,format=hang,font=small}
% \NewDocumentCommand \im { o o m o } {%
%   % \begin{framed}
%   \IfNoValueTF{#2} {%
%     \IfNoValueTF{#1} {%
%       \inputminted[fontsize=\small,xleftmargin=0.25in,linenos,breaklines]%
%         {<++>}{#3}%
%     } {%
%       \inputminted[fontsize=\small,xleftmargin=0.25in,linenos,breaklines,%
%         firstline=#1,lastline=#1]{<++>}{#3}%
%     }%
%   } {%
%     \inputminted[fontsize=\small,xleftmargin=0.25in,linenos,breaklines,%
%       firstline=#1,lastline=#2]{<++>}{#3}%
%     } \IfValueT{#4}{\widecaps[listing]{#4}}%
%   % \end{framed}
% }
% \NewDocumentCommand \gob { m o o m o } {%
% % \begin{framed}
%   \IfNoValueTF{#3} {%
%     \IfNoValueTF{#2} {%
%       \inputminted[fontsize=\small,xleftmargin=1.25in,linenos,breaklines,%
%         gobble=#1]{<++>}{#4}%
%     } {%
%       \inputminted[fontsize=\small,xleftmargin=0.25in,linenos,breaklines,%
%         gobble=#1,firstline=#2,lastline=#2]{<++>}{#4}%
%     }%
%   } {%
%     \inputminted[fontsize=\small,xleftmargin=0.25in,linenos,breaklines,%
%       gobble=#1,firstline=#2,lastline=#3]{<++>}{#4}%
%   } \IfValueT{#5}{\widecaps[listing]{#5}}%
% % \end{framed}
% }
%
% \DeclareFloatingEnvironment[name=Result]{result}
% \captionsetup[result]{aboveskip=0pt,belowskip=0pt,format=hang,font=small}
% \NewDocumentCommand \res { o o m o } {%
% % \begin{framed}
%   \IfNoValueTF{#2} {%
%     \IfNoValueTF{#1} {%
%       \lstinputlisting[basicstyle=\ttfamily\small,breaklines,%
%         columns=fullflexible]{#3}%
%     } {%
%       \lstinputlisting[basicstyle=\ttfamily\small,breaklines,%
%         columns=fullflexible,firstline=#1,lastline=#1]{#3}%
%     }%
%   } {%
%     \lstinputlisting[basicstyle=\ttfamily\small,breaklines,%
%       columns=fullflexible,firstline=#1,lastline=#2]{#3}%
%   }%
% % \end{framed}
%   \IfValueT{#4}{\widecaps[result]{#4}}%
% }


\begin{document}

  \title{Lab 3 : Branch Predictor Integration}
  \author{2018-17515 김형준}
  \date{30 May 2022}
  \maketitle
  \thispagestyle{empty}

Branch Prediction은 IF 단계에서 새로 입력된 PC를 기반으로, 향후 어떠한 방향으로
프로그램이 진행될 지를 예상하는 과정이다. 하나의 PC는 변하지 않는 하나의
instruction과 대응되므로, 기존 PC값을 분석해 향후 해당 PC값이 다시 입력될 때
더 정확한 예측을 할 수 있다. 본 과제에서는 2-bit Gshare branch predictor 기법을
활용해 branch prediction을 성공적으로 구현했으며, 기존에 비해 core cycle 성능과
branch prediction 정확도가 증가했음을 알 수 있다 :

\begin{center}
  \begin{tabular}{ *2{C{.48\linewidth} } }
    % This file was created with tikzplotlib v0.10.1.
\begin{tikzpicture}

\definecolor{darkgray176}{RGB}{176,176,176}
\definecolor{lightgray204}{RGB}{204,204,204}
\definecolor{teal}{RGB}{0,128,128}

\begin{axis}[
legend cell align={left},
legend style={
  fill opacity=0.8,
  draw opacity=1,
  text opacity=1,
  at={(0.03,0.97)},
  anchor=north west,
  draw=lightgray204
},
tick align=outside,
tick pos=left,
title={Core Cycle Record},
x grid style={darkgray176},
xlabel={Benchmark Workloads},
xmin=-0.635, xmax=5.635,
xticklabel style={rotate=90},
xtick={0,1,2,3,4,5},
xticklabels={
  \texttt{bst\_array},
  \texttt{fibo},
  \texttt{matmul},
  \texttt{prime\_fact},
  \texttt{quicksort},
  \texttt{spmv}
},
y grid style={darkgray176},
ylabel={Core Cycles Processed},
ymin=0, ymax=44314.2,
ytick style={color=black}
]
\draw[draw=none,fill=blue] (axis cs:-0.35,0) rectangle (axis cs:-0.05,8745);
\addlegendimage{ybar,ybar legend,draw=none,fill=blue}
\addlegendentry{Branch Prediction 미적용}

\draw[draw=none,fill=blue] (axis cs:0.65,0) rectangle (axis cs:0.95,1436);
\draw[draw=none,fill=blue] (axis cs:1.65,0) rectangle (axis cs:1.95,19536);
\draw[draw=none,fill=blue] (axis cs:2.65,0) rectangle (axis cs:2.95,16313);
\draw[draw=none,fill=blue] (axis cs:3.65,0) rectangle (axis cs:3.95,31504);
\draw[draw=none,fill=blue] (axis cs:4.65,0) rectangle (axis cs:4.95,42204);
\draw[draw=none,fill=teal] (axis cs:0.05,0) rectangle (axis cs:0.35,7957);
\addlegendimage{ybar,ybar legend,draw=none,fill=teal}
\addlegendentry{Branch Prediction 적용}

\draw[draw=none,fill=teal] (axis cs:1.05,0) rectangle (axis cs:1.35,1404);
\draw[draw=none,fill=teal] (axis cs:2.05,0) rectangle (axis cs:2.35,17760);
\draw[draw=none,fill=teal] (axis cs:3.05,0) rectangle (axis cs:3.35,16823);
\draw[draw=none,fill=teal] (axis cs:4.05,0) rectangle (axis cs:4.35,28962);
\draw[draw=none,fill=teal] (axis cs:5.05,0) rectangle (axis cs:5.35,34102);
\end{axis}

\end{tikzpicture}

    \widecaps{Branch Prediction CPU 성능 개선} &
    % This file was created with tikzplotlib v0.10.1.
\begin{tikzpicture}

\definecolor{darkgray176}{RGB}{176,176,176}
\definecolor{lightgray204}{RGB}{204,204,204}
\definecolor{magenta}{RGB}{255,0,255}

\begin{axis}[
legend cell align={left},
legend style={
  fill opacity=0.8,
  draw opacity=1,
  text opacity=1,
  at={(0.03,0.97)},
  anchor=north west,
  draw=lightgray204
},
tick align=outside,
tick pos=left,
title={Branch Predictor Accuracy},
x grid style={darkgray176},
xlabel={Benchmark Workloads},
xmin=-0.635, xmax=5.635,
xticklabel style={rotate=90},
xtick={0,1,2,3,4,5},
xticklabels={
  \texttt{bst\_array},
  \texttt{fibo},
  \texttt{matmul},
  \texttt{prime\_fact},
  \texttt{quicksort},
  \texttt{spmv}
},
y grid style={darkgray176},
ylabel={Successful Branch Predictions},
ymin=0, ymax=8513.4,
ytick style={color=black}
]
\draw[draw=none,fill=red] (axis cs:-0.35,0) rectangle (axis cs:-0.05,730);
\addlegendimage{ybar,ybar legend,draw=none,fill=red}
\addlegendentry{Branch Prediction 미적용}

\draw[draw=none,fill=red] (axis cs:0.65,0) rectangle (axis cs:0.95,210);
\draw[draw=none,fill=red] (axis cs:1.65,0) rectangle (axis cs:1.95,1417);
\draw[draw=none,fill=red] (axis cs:2.65,0) rectangle (axis cs:2.95,2692);
\draw[draw=none,fill=red] (axis cs:3.65,0) rectangle (axis cs:3.95,2665);
\draw[draw=none,fill=red] (axis cs:4.65,0) rectangle (axis cs:4.95,4061);
\draw[draw=none,fill=magenta] (axis cs:0.05,0) rectangle (axis cs:0.35,1121);
\addlegendimage{ybar,ybar legend,draw=none,fill=magenta}
\addlegendentry{Branch Prediction 적용}

\draw[draw=none,fill=magenta] (axis cs:1.05,0) rectangle (axis cs:1.35,227);
\draw[draw=none,fill=magenta] (axis cs:2.05,0) rectangle (axis cs:2.35,2123);
\draw[draw=none,fill=magenta] (axis cs:3.05,0) rectangle (axis cs:3.35,3241);
\draw[draw=none,fill=magenta] (axis cs:4.05,0) rectangle (axis cs:4.35,3753);
\draw[draw=none,fill=magenta] (axis cs:5.05,0) rectangle (axis cs:5.35,8108);
\end{axis}

\end{tikzpicture}

    \widecaps{Branch Prediction 정확도 증가} \\
  \end{tabular}
\end{center}

특히 여러 시행에서 예상보다 높은 빈도로 Branch Prediction이 성공함을 확인할 수
있었는데, 이는 근본적으로 Pipeline Architecture의 Hazard Control 중 다음 PC값,
즉 \verb+next_PC+값이 Branch의 Target Address, 즉 \verb+ex_pc_target+값과
동일할 때 flush를 진행하지 않는, 사실상의 taken 처리를 해주기에 발생하는
현상으로 판단된다.

\verb+branch_hardware.v+ 모듈은 \verb+branch_target_buffer.v+와
\verb+gshare.v+의 입출력 관계를 정립하는 역할을 하며, \\
\verb+branch_target_buffer.v+와 \verb+gshare.v+에서는 각각 branch target
buffer와 Gshare 메모리를 선언, 해당 배열에 접근하는 방식과 배열 내 저장된
데이터를 새로고침하는 방법을 정의했다.

\end{document}
