% MATH
% Better Vector Notation
\makeatletter
\newlength\xvec@height%
\newlength\xvec@depth%
\newlength\xvec@width%
\newcommand{\xvec}[2][]{%
  \ifmmode%
    \settoheight{\xvec@height}{$#2$}%
    \settodepth{\xvec@depth}{$#2$}%
    \settowidth{\xvec@width}{$#2$}%
  \else%
    \settoheight{\xvec@height}{#2}%
    \settodepth{\xvec@depth}{#2}%
    \settowidth{\xvec@width}{#2}%
  \fi%
  \def\xvec@arg{#1}%
  \def\xvec@dd{:}%
  \def\xvec@d{.}%
  \raisebox{.2ex}{\raisebox{\xvec@height}{\rlap{%
    \kern.05em%  (Because left edge of drawing is at .05em)
    \begin{tikzpicture}[scale=1]
    \pgfsetroundcap
    \draw (.05em,0)--(\xvec@width-.05em,0);
    \draw (\xvec@width-.05em,0)--(\xvec@width-.15em, .075em);
    \draw (\xvec@width-.05em,0)--(\xvec@width-.15em,-.075em);
    \ifx\xvec@arg\xvec@d%
      \fill(\xvec@width*.45,.5ex) circle (.5pt);%
    \else\ifx\xvec@arg\xvec@dd%
      \fill(\xvec@width*.30,.5ex) circle (.5pt);%
      \fill(\xvec@width*.65,.5ex) circle (.5pt);%
    \fi\fi%
    \end{tikzpicture}%
  }}}%
  #2%
}
\makeatother

\let\stdvec\vec
% \renewcommand{\vec}[1]{\xvec[]{#1}}

% --- Define \dvec and \ddvec for dotted and double-dotted vectors.
\newcommand{\dvec}[1]{\xvec[.]{#1}}
\newcommand{\ddvec}[1]{\xvec[:]{#1}}

% Personalised macros
\DeclareRobustCommand{\[}{\begin{equation}}
\DeclareRobustCommand{\]}{\end{equation}}
\let\myvec\vec
\renewcommand{\vec}[1]{\mathbf{#1}} % Vector with varying height
\newcommand{\hivec}[2][b]{\xvec[]{\vphantom{#1}#2}} % Vector with consistent height
\newcommand{\abs}[1]{\ensuremath{\lvert #1 \rvert}}
\newcommand{\round}[1]{\ensuremath{\partial #1}}
\newcommand{\D}[1]{\ensuremath{\Delta #1}}
\NewDocumentCommand \len { o m } {
  \IfNoValueTF{#1} { \abs{\vec{#2}} } { \abs{\xvec[]{\vphantom{#1}#2}} }
}
\newcommand{\mat}[1]{\mathbf{#1}} % Matrix Notation
\NewDocumentCommand \pafr { o m o m o } {%
  \IfNoValueTF{#5} {%
    \IfNoValueTF{#1} {%
      \IfNoValueTF{#3} {%
        \ensuremath{\frac{\round{#2}}{\round{#4}}}
      } {%
        \ensuremath{\frac{\round{#2}}{#3\,\round{#4}}}
      }
    } {%
      \IfNoValueTF{#3} {%
        \ensuremath{\frac{#1\,\round{#2}}{\round{#4}}}
      } {%
        \ensuremath{\frac{#1\,\round{#2}}{#3\,\round{#4}}}
      }
    }
  } {
    \IfNoValueTF{#1} {%
      \IfNoValueTF{#3} {%
        \ensuremath{\frac{\partial\sp{#5}#2}{\partial #4\sp{#5}}}
      } {%
        \ensuremath{\frac{\partial\sp{#5}#2}{#3\,\partial #4\sp{#5}}}
      }
    } {%
      \IfNoValueTF{#5} {%
        \ensuremath{\frac{#1\,\partial\sp{#5}#2}{\partial #4\sp{#5}}}
      } {%
        \ensuremath{\frac{#1\,\partial\sp{#5}#2}{#3\,\partial #4\sp{#5}}}
      }
    }
  }
}
\NewDocumentCommand \difr { o m o m o } {%
  \IfNoValueTF{#5} {%
    \IfNoValueTF{#1} {%
      \IfNoValueTF{#3} {%
        \ensuremath{\frac{d #2}{d #4}}
      } {%
        \ensuremath{\frac{d #2}{#3\,d #4}}
      }
    } {%
      \IfNoValueTF{#3} {%
        \ensuremath{\frac{#1\,d #2}{d #4}}
      } {%
        \ensuremath{\frac{#1\,d #2}{#3\,d #4}}
      }
    }
  } {
    \IfNoValueTF{#1} {%
      \IfNoValueTF{#3} {%
        \ensuremath{\frac{d\sp{#5} #2}{d #4\sp{#5}}}
      } {%
        \ensuremath{\frac{d\sp{#5} #2}{#3\,d #4\sp{#5}}}
      }
    } {%
      \IfNoValueTF{#5} {%
        \ensuremath{\frac{#1\,d\sp{#5} #2}{d #4\sp{#5}}}
      } {%
        \ensuremath{\frac{#1\,d\sp{#5} #2}{#3\,d #4\sp{#5}}}
      }
    }
  }
}
\NewDocumentCommand \VTP { o m m m } {%
  \IfNoValueTF{#1} {%
    \ensuremath{\vec{#2}\times\vec{#3}\times\vec{#4}}
  } {
    \ensuremath{\hivec[#1]{#2}\times\hivec[#1]{#3}\times\hivec[#1]{#4}}
  }
}
\NewDocumentCommand \STP { o m m m } {%
  \IfNoValueTF{#1} {%
    \ensuremath{\vec{#2}\wedge\vec{#3}\wedge\vec{#4}}
  } {
    \ensuremath{\hivec[#1]{#2}\wedge\hivec[#1]{#3}\wedge\hivec[#1]{#4}}
  }
}
\newcommand{\vecf}[2]{\ensuremath\vec{#1}\left(#2\right)}
\newcommand{\grad}[1]{\ensuremath\nabla #1}
\newcommand{\dive}[1]{\ensuremath\nabla\DOT{#1}}
\newcommand{\curl}[1]{\ensuremath\nabla\times{#1}}
\newcommand{\lapl}[1]{\ensuremath\nabla\sp{2}#1}
\makeatletter
\newcommand*{\DOT}{}% Check if undefined
\DeclareRobustCommand*{\DOT}{%
  \mathbin{\mathpalette\DOT@{}}%
}
\newcommand*{\DOT@scalefactor}{.5}
\newcommand*{\DOT@widthfactor}{1.15}
\newcommand*{\DOT@}[2]{%
  % #1: math style
  % #2: unused
  \sbox0{$#1\vcenter{}$}% math axis
  \sbox2{$#1\cdot\m@th$}%
  \hbox to \DOT@widthfactor\wd2{%
  \hfil
  \raise\ht0\hbox{%
    \scalebox{\DOT@scalefactor}{%
    \lower\ht0\hbox{$#1\bullet\m@th$}%
    }%
  }%
  \hfil
  }%
}
\makeatother
\NewDocumentCommand \uvec { o } {%
  \IfValueTF{#1}{\ensuremath{\vec{a}\sb{#1}}}%
    {\ensuremath{\vec{\mathrm{a}}}}
}

% % CODE
% \NewDocumentCommand \mi { o } {%
%   \IfValueTF{#1} {%
%     \mintinline[#1]{<++>}%
%   } {%
%     \mintinline{<++>}%
%   }%
% }
%
% \SetupFloatingEnvironment{listing}{name=Code}
% \captionsetup[listing]{aboveskip=0pt,belowskip=0pt,format=hang,font=small}
% \NewDocumentCommand \im { o o m o } {%
%   % \begin{framed}
%   \IfNoValueTF{#2} {%
%     \IfNoValueTF{#1} {%
%       \inputminted[fontsize=\small,xleftmargin=0.25in,linenos,breaklines]%
%         {<++>}{#3}%
%     } {%
%       \inputminted[fontsize=\small,xleftmargin=0.25in,linenos,breaklines,%
%         firstline=#1,lastline=#1]{<++>}{#3}%
%     }%
%   } {%
%     \inputminted[fontsize=\small,xleftmargin=0.25in,linenos,breaklines,%
%       firstline=#1,lastline=#2]{<++>}{#3}%
%     } \IfValueT{#4}{\widecaps[listing]{#4}}%
%   % \end{framed}
% }
% \NewDocumentCommand \gob { m o o m o } {%
% % \begin{framed}
%   \IfNoValueTF{#3} {%
%     \IfNoValueTF{#2} {%
%       \inputminted[fontsize=\small,xleftmargin=1.25in,linenos,breaklines,%
%         gobble=#1]{<++>}{#4}%
%     } {%
%       \inputminted[fontsize=\small,xleftmargin=0.25in,linenos,breaklines,%
%         gobble=#1,firstline=#2,lastline=#2]{<++>}{#4}%
%     }%
%   } {%
%     \inputminted[fontsize=\small,xleftmargin=0.25in,linenos,breaklines,%
%       gobble=#1,firstline=#2,lastline=#3]{<++>}{#4}%
%   } \IfValueT{#5}{\widecaps[listing]{#5}}%
% % \end{framed}
% }
%
% \DeclareFloatingEnvironment[name=Result]{result}
% \captionsetup[result]{aboveskip=0pt,belowskip=0pt,format=hang,font=small}
% \NewDocumentCommand \res { o o m o } {%
% % \begin{framed}
%   \IfNoValueTF{#2} {%
%     \IfNoValueTF{#1} {%
%       \lstinputlisting[basicstyle=\ttfamily\small,breaklines,%
%         columns=fullflexible]{#3}%
%     } {%
%       \lstinputlisting[basicstyle=\ttfamily\small,breaklines,%
%         columns=fullflexible,firstline=#1,lastline=#1]{#3}%
%     }%
%   } {%
%     \lstinputlisting[basicstyle=\ttfamily\small,breaklines,%
%       columns=fullflexible,firstline=#1,lastline=#2]{#3}%
%   }%
% % \end{framed}
%   \IfValueT{#4}{\widecaps[result]{#4}}%
% }
